\documentclass[a4paper,notitlepage,oneside]{report}

\usepackage[cm-default]{fontspec}
\usepackage{fullpage}
\usepackage[utf8]{inputenc}
\usepackage[bulgarian]{babel}
\usepackage{xecyr}
\usepackage{amsmath}
\usepackage{color}
\usepackage{xcolor}
\usepackage{listings}
\usepackage{multicol}

\defaultfontfeatures{Mapping=tex-text,Scale=MatchLowercase}
\setmainfont{Linux Libertine O}
\setsansfont{Linux Biolinum O}
\setmonofont{Inconsolata}
\lstset{keywordstyle=\color{Blue},commentstyle=\color{teal},
        stringstyle=\color{brown},basicstyle=\footnotesize\ttfamily,
        title=\lstname}

\title{Домашно упражнение №1}
\author{Петко Борджуков (Ф№ 61322)}
\date{}

\begin{document}
\maketitle
\setcounter{secnumdepth}{0}
\section{Задача 1}

За да решим задачата, дефинираме класовете \emph{HadamardMatrix} и
\emph{GolayCode}. Класът \emph{HadamardMatrix} няма полета и предлага на
потребителя два статични метода за създаване на матрица на Адамар --
\emph{\#paley(order)} и \emph{\#sylvester(order)}.

Класът \emph{GolayCode} има полета \emph{dimension} и \emph{matrix}, съдържащи
съответно размерността на кода и матрицата му. Инстанция на този клас може да се
създаде, чрез подаване на желана размерност чрез извикване на
\emph{GolayCode\#new(dimension)}.

\subsection{Отпечатване на матрица на Адамар от ред n, съставена по метода на
  Пейли}
Методът \emph{HadamardMatrix\#paley(order)} съставя матрица по метода на Пейли,
както е описан в Двоични шумозащитни кодове, 2004, Е. Великова-Бандова.

При подаване на ред, матрица от който не може да бъде създадена по този метод,
програмата връща грешка:
\begin{lstlisting}
[38] pry(main)> HadamardMatrix.paley 18
RuntimeError: The given order 18 is not valid
\end{lstlisting}

При подаване на ред, матрица от който може да бъде създадена по метода на Пейли,
програмата връща инстанция на класа \emph{Matrix}, която отговаря на условията,
изисквани, за да бъде наречена Адамарова:

\begin{multicols}{2}

\begin{lstlisting}
[43] pry(main)> m = HadamardMatrix.paley(12).to_a
=> [
 [ 1, 1, 1, 1, 1, 1, 1, 1, 1, 1, 1, 1],
 [ 1,-1, 1,-1, 1, 1, 1,-1,-1,-1, 1,-1],
 [ 1,-1,-1, 1,-1, 1, 1, 1,-1,-1,-1, 1],
 [ 1, 1,-1,-1, 1,-1, 1, 1, 1,-1,-1,-1],
 [ 1,-1, 1,-1,-1, 1,-1, 1, 1, 1,-1,-1],
 [ 1,-1,-1, 1,-1,-1, 1,-1, 1, 1, 1,-1],
 [ 1,-1,-1,-1, 1,-1,-1, 1,-1, 1, 1, 1],
 [ 1, 1,-1,-1,-1, 1,-1,-1, 1,-1, 1, 1],
 [ 1, 1, 1,-1,-1,-1, 1,-1,-1, 1,-1, 1],
 [ 1, 1, 1, 1,-1,-1,-1, 1,-1,-1, 1,-1],
 [ 1,-1, 1, 1, 1,-1,-1,-1, 1,-1,-1, 1],
 [ 1, 1,-1, 1, 1, 1,-1,-1,-1, 1,-1,-1]
]
\end{lstlisting}

\begin{lstlisting}
[52] pry(main)> m.each {|row| row.collect!\
   {|val| val == -1 ? "*" : " "}; p row.join ' '}
"                       "
"  *   *       * * *   *"
"  * *   *       * * *  "
"    * *   *       * * *"
"  *   * *   *       * *"
"  * *   * *   *       *"
"  * * *   * *   *      "
"    * * *   * *   *    "
"      * * *   * *   *  "
"        * * *   * *   *"
"  *       * * *   * *  "
"    *       * * *   * *"

\end{lstlisting}

\end{multicols}

\subsection{Отпечатване на матрица на Адамар от ред n, съставена по метода на
  Силвестър}
Методът \emph{HadamardMatrix\#sylvester(order)} рекурсивно съставя матрица, чрез
удвояване на реда.

При подаване на ред, матрица от който не може да бъде създадена по този метод,
програмата връща грешка:
\begin{lstlisting}
[60] pry(main)> HadamardMatrix.sylvester 12
RuntimeError: The given order 12 is not valid
\end{lstlisting}

При подаване на валиден ред, методът връща инстанция на класа \emph{Matrix},
която отговаря на условията, изисквани, за да може да бъде наречена Адамарова.
\begin{lstlisting}
[63] pry(main)> m = HadamardMatrix.sylvester(32).to_a
[65] pry(main)> m.each {|row| row.collect! {|val| val == -1 ? "*" : " "}; p row.join ' '}
"                                                               "
"  *   *   *   *   *   *   *   *   *   *   *   *   *   *   *   *"
"    * *     * *     * *     * *     * *     * *     * *     * *"
"  * *     * *     * *     * *     * *     * *     * *     * *  "
"        * * * *         * * * *         * * * *         * * * *"
"  *   * *   *     *   * *   *     *   * *   *     *   * *   *  "
"    * * * *         * * * *         * * * *         * * * *    "
"  * *   *     *   * *   *     *   * *   *     *   * *   *     *"
"                * * * * * * * *                 * * * * * * * *"
"  *   *   *   * *   *   *   *     *   *   *   * *   *   *   *  "
"    * *     * * * *     * *         * *     * * * *     * *    "
"  * *     * *   *     * *     *   * *     * *   *     * *     *"
"        * * * * * * * *                 * * * * * * * *        "
"  *   * *   *   *   *     *   *   *   * *   *   *   *     *   *"
"    * * * *     * *         * *     * * * *     * *         * *"
"  * *   *     * *     *   * *     * *   *     * *     *   * *  "
"                                * * * * * * * * * * * * * * * *"
"  *   *   *   *   *   *   *   * *   *   *   *   *   *   *   *  "
"    * *     * *     * *     * * * *     * *     * *     * *    "
"  * *     * *     * *     * *   *     * *     * *     * *     *"
"        * * * *         * * * * * * * *         * * * *        "
"  *   * *   *     *   * *   *   *   *     *   * *   *     *   *"
"    * * * *         * * * *     * *         * * * *         * *"
"  * *   *     *   * *   *     * *     *   * *   *     *   * *  "
"                * * * * * * * * * * * * * * * *                "
"  *   *   *   * *   *   *   *   *   *   *   *     *   *   *   *"
"    * *     * * * *     * *     * *     * *         * *     * *"
"  * *     * *   *     * *     * *     * *     *   * *     * *  "
"        * * * * * * * *         * * * *                 * * * *"
"  *   * *   *   *   *     *   * *   *     *   *   *   * *   *  "
"    * * * *     * *         * * * *         * *     * * * *    "
"  * *   *     * *     *   * *   *     *   * *     * *   *     *"
\end{lstlisting}

\begin{lstlisting}
[66] pry(main)> m = HadamardMatrix.sylvester(64)
[69] pry(main)> m * m.transpose == Matrix.scalar(64, 64)
=> true
\end{lstlisting}

\subsection{Съставяне на удължен код на Голей}
Методът \emph{GolayCode\#new(dimension)} съставя матрица на Адамар от ред
\emph{dimension} и с нейна помощ образува матрицата на разширения код на Голей с
размерност \emph{dimension}.

\begin{lstlisting}
[2] pry(main)> g = GolayCode.new 12
[3] pry(main)> g.matrix.to_a
=> [
 [1, 0, 0, 0, 0, 0, 0, 0, 0, 0, 0, 0, 1, 1, 1, 1, 1, 1, 1, 1, 1, 1, 1, 0],
 [0, 1, 0, 0, 0, 0, 0, 0, 0, 0, 0, 0, 1, 1, 0, 1, 1, 1, 0, 0, 0, 1, 0, 1],
 [0, 0, 1, 0, 0, 0, 0, 0, 0, 0, 0, 0, 0, 1, 1, 0, 1, 1, 1, 0, 0, 0, 1, 1],
 [0, 0, 0, 1, 0, 0, 0, 0, 0, 0, 0, 0, 1, 0, 1, 1, 0, 1, 1, 1, 0, 0, 0, 1],
 [0, 0, 0, 0, 1, 0, 0, 0, 0, 0, 0, 0, 0, 1, 0, 1, 1, 0, 1, 1, 1, 0, 0, 1],
 [0, 0, 0, 0, 0, 1, 0, 0, 0, 0, 0, 0, 0, 0, 1, 0, 1, 1, 0, 1, 1, 1, 0, 1],
 [0, 0, 0, 0, 0, 0, 1, 0, 0, 0, 0, 0, 0, 0, 0, 1, 0, 1, 1, 0, 1, 1, 1, 1],
 [0, 0, 0, 0, 0, 0, 0, 1, 0, 0, 0, 0, 1, 0, 0, 0, 1, 0, 1, 1, 0, 1, 1, 1],
 [0, 0, 0, 0, 0, 0, 0, 0, 1, 0, 0, 0, 1, 1, 0, 0, 0, 1, 0, 1, 1, 0, 1, 1],
 [0, 0, 0, 0, 0, 0, 0, 0, 0, 1, 0, 0, 1, 1, 1, 0, 0, 0, 1, 0, 1, 1, 0, 1],
 [0, 0, 0, 0, 0, 0, 0, 0, 0, 0, 1, 0, 0, 1, 1, 1, 0, 0, 0, 1, 0, 1, 1, 1],
 [0, 0, 0, 0, 0, 0, 0, 0, 0, 0, 0, 1, 1, 0, 1, 1, 1, 0, 0, 0, 1, 0, 1, 1]
]
[7] pry(main)> g.matrix.to_a.each {|row| \
        row.collect! {|val| val == 1 ? '*' : " "}; p row.join ' '}
"*                       * * * * * * * * * * *  "
"  *                     * *   * * *       *   *"
"    *                     * *   * * *       * *"
"      *                 *   * *   * * *       *"
"        *                 *   * *   * * *     *"
"          *                 *   * *   * * *   *"
"            *                 *   * *   * * * *"
"              *         *       *   * *   * * *"
"                *       * *       *   * *   * *"
"                  *     * * *       *   * *   *"
"                    *     * * *       *   * * *"
"                      * *   * * *       *   * *"
\end{lstlisting}


\subsection*{Изходен код}
\lstinputlisting[language=Ruby]{hadamard_matrix.rb}
\lstinputlisting[language=Ruby]{golay_code.rb}

\subsection*{Спецификация на тестовете}

Изходният код, изведен по-горе, удовлетворява следните тестове:

\begin{lstlisting}
GolayCode
  can be instantiated only by passing a valid code dimension
  #matrix
    returns the generator matrix of the code
  #minimal_distance
    returns the minimal distance of the code

HadamardMatrix
  cannot be instantiated
  #paley
    behaves like a Hadamard matrix constructor
      when passed an invalid order
        raises an error
      when passed a valid order
        does not raise an error
        constructs a matrix
        constructs a square matrix
        constructs a matrix that contains only 1s and -1s
        constructs a matrix that contains only mutually orthogonal rows
        constructs a matrix that matches the given reference Hadamard matrix
  #sylvester
    behaves like a Hadamard matrix constructor
      when passed an invalid order
        raises an error
      when passed a valid order
        does not raise an error
        constructs a matrix
        constructs a square matrix
        constructs a matrix that contains only 1s and -1s
        constructs a matrix that contains only mutually orthogonal rows
        constructs a matrix that matches the given reference Hadamard matrix

\end{lstlisting}
\section{Задача 2}
Нека имаме кода $C$ със следната пораждаща матрица:
\[
G = 
\begin{pmatrix}
  1 & 0 & 1 & 1 & 0 & 0\\
  1 & 1 & 0 & 1 & 0 & 1\\
  1 & 0 & 1 & 0 & 1 & 1
\end{pmatrix}
\]
\begin{itemize}
  \item Размерността на кода $C$ е $k=3$; минималното му разстояние е $d(C)=wt(C) = 3$.
  \item Стандартната таблица на Слепян за кода $C$:
    \begin{table}[h]
      \begin{tabular}{ | c | c c c c c c c |}
        \hline
          (0,0,0,0,0,0) & (0,0,0,1,1,1) & (0,1,1,0,0,1) & (0,1,1,1,1,0) & (1,0,1,0,1,1) & (1,0,1,1,0,0) & (1,1,0,0,1,0) & (1,1,0,1,0,1)\\ 
          (0,0,0,0,0,1) & (0,0,0,1,1,0) & (0,1,1,0,0,0) & (0,1,1,1,1,1) & (1,0,1,0,1,0) & (1,0,1,1,0,1) & (1,1,0,0,1,1) & (1,1,0,1,0,0)\\
          (0,0,0,0,1,0) & (0,0,0,1,0,1) & (0,1,1,0,1,1) & (0,1,1,1,0,0) & (1,0,1,0,0,1) & (1,0,1,1,1,0) & (1,1,0,0,0,0) & (1,1,0,1,1,1)\\
          (0,0,0,1,0,0) & (0,0,0,0,1,1) & (0,1,1,1,0,1) & (0,1,1,0,1,0) & (1,0,1,1,1,1) & (1,0,1,0,0,0) & (1,1,0,1,1,0) & (1,1,0,0,0,1)\\
          (0,0,1,0,0,0) & (0,0,1,1,1,1) & (0,1,0,0,0,1) & (0,1,0,1,1,0) & (1,0,0,0,1,1) & (1,0,0,1,0,0) & (1,1,1,0,1,0) & (1,1,1,1,0,1)\\
          (0,1,0,0,0,0) & (0,1,0,1,1,1) & (0,0,1,0,0,1) & (0,0,1,1,1,0) & (1,1,1,0,1,1) & (1,1,1,1,0,0) & (1,0,0,0,1,0) & (1,0,0,1,0,1)\\
          (1,0,0,0,0,0) & (1,0,0,1,1,1) & (1,1,1,0,0,1) & (1,1,1,1,1,0) & (0,0,1,0,1,1) & (0,0,1,1,0,0) & (0,1,0,0,1,0) & (0,1,0,1,0,1)\\ \hline
          (0,0,1,0,1,0) & (0,0,1,1,0,1) & (0,1,0,0,1,1) & (0,1,0,1,0,0) & (1,0,0,0,0,1) & (1,0,0,1,1,0) & (1,1,1,0,0,0) & (1,1,1,1,1,1)\\ \hline
      \end{tabular}
    \end{table}
    
    \begin{itemize}
    \item Векторът (0,1,1,0,1,0) се декодира до думата (0,1,1,1,1,0) с грешка
      (0,0,0,1,0,0).
    \item Векторът (0,1,0,1,0,1) се декодира до думата (1,1,0,1,0,1) с грешка
      (1,0,0,0,0,0).
    \item Векторът (1,1,1,0,0,0) не може да се декодира еднозначно до кодова
      дума.
    \end{itemize}
\end{itemize}

\section{Задача 3}
Ако двоичният линеен код $C$ с нотация $[n, k]$ съдържа думи с нечетно тегло, то
те са точно $2^{k-1}$ на брой.
\end{document}
