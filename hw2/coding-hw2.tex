\documentclass[a4paper,notitlepage,oneside]{report}

\usepackage[cm-default]{fontspec}
\usepackage[utf8]{inputenc}
\usepackage[bulgarian]{babel}
\usepackage{xecyr}
\usepackage{amsmath}
\usepackage{color}
\usepackage{xcolor}
\usepackage{listings}

\defaultfontfeatures{Mapping=tex-text,Scale=MatchLowercase}
\setmainfont{Linux Libertine O}
\setsansfont{Linux Biolinum O}
\setmonofont{Inconsolata}
\lstset{keywordstyle=\color{blue},commentstyle=\color{teal},stringstyle=\color{brown},basicstyle=\footnotesize\ttfamily,title=\lstname}

\title{Домашно упражнение №2}
\author{Петко Борджуков (Ф№ 61322)}
\date{}

\begin{document}
\maketitle
\setcounter{secnumdepth}{0}
\section{Задача 1}

За да решим задачата, дефинираме обект \emph{Polynom} с полета \emph{coefficients}, \emph{power} и \emph{truth\_table}, които съдържат съответно вектора от коефициенти, степента и вярностната таблица на полинома. Инстанция на така зададения обект може да бъде създадена по един от два начина -- чрез метода \emph{Polynom\#[]=(*coefficients)} по коефициенти, или чрез метода \emph{Polynom\#truth\_vector(*vector)} по вярностен вектор.

\subsection{Вярностна таблица на полином по вектор от коефициенти}
Вярностната таблица на полином, създаден чрез подаване на вектор от коефициенти, попълваме чрез заместване. Методът \emph{Polynom\#truth\_table} връща хештаблица от вида \emph{вектор от стойности на аргументите    => стойност}.

\begin{lstlisting}
007 > require './polynom.rb'
008 > p = Polynom[1, 0, 0, 1, 1, 0, 0, 0]
009 > p.truth_table.each {|key, value| p "%s    => %s" %[key, value]}
"[0, 0, 0]    => 1"
"[1, 0, 0]    => 1"
"[0, 1, 0]    => 1"
"[0, 0, 1]    => 0"
"[1, 1, 0]    => 0"
"[1, 0, 1]    => 0"
"[0, 1, 1]    => 0"
"[1, 1, 1]    => 1"
\end{lstlisting}


\subsection{Коефициенти на полином по вярностен вектор}
Векторът от коефициенти на полином, създаден чрез подаване на вярностен вектор, намираме по начина за намиране на полином на Жегалкин на булева функция чрез решаване на система, описан в К. Манев - „Увод в дискретната математика“. Методът \emph{Polynom\#coefficients} връща вектор от коефициентите на полинома.

\begin{lstlisting}
011 > require './polynom.rb'
012 > p = Polynom.truth_vector(1, 1, 1, 0, 0, 0, 0, 1) 
013 > p.coefficients
    => Vector[1, 0, 0, 1, 1, 0, 0, 0] 
014 >
\end{lstlisting}

\section{Задача 2}
За решението на тази задача, дефинираме клас \emph{RM} с полета \emph{r}, \emph{m} и \emph{matrix}, съдържащи съответно редът, дължината и пораждащата матрица на кода. Инстанция на така зададения клас може да се създаде чрез извикване на метода му \emph{RM\#new(r, m)}.

\subsection{Извеждане на всички кодове с дължина $m$}
При зададено $m=10$, изходът от програмата е във вида $RM(r_i,m)$, $\left[n, k, d \right]$, corrects $t$.
\begin{lstlisting}
020 > require './rm.rb'
    => false 
021 > m = 10
    => 10 
022 > m.times do |r|
023?>   rm = RM.new r,m
024?>   p "%s, %s, corrects %s" %[rm, rm.notation, rm.corrects] 
025?> end
"RM(0, 10), [1024, 1, 1024], corrects 511"
"RM(1, 10), [1024, 11, 512], corrects 255"
"RM(2, 10), [1024, 56, 256], corrects 127"
"RM(3, 10), [1024, 176, 128], corrects 63"
"RM(4, 10), [1024, 386, 64], corrects 31"
"RM(5, 10), [1024, 638, 32], corrects 15"
"RM(6, 10), [1024, 848, 16], corrects 7"
"RM(7, 10), [1024, 968, 8], corrects 3"
"RM(8, 10), [1024, 1013, 4], corrects 1"
"RM(9, 10), [1024, 1023, 2], corrects 0"
    => 10 
026 > 
\end{lstlisting}

\subsection{Извеждане на параметрите на код по устойчивост на грешки}
Методите \emph{RM\#correcting(t, min\_d)} и \emph{RM\#detecting(t, min\_d)}, където $t$ е брой грешки, а $min\_d$ -- минимална размерност на кода, изчисляват реда и дължината на най-оптималния код, който покрива изискванията и връщат инстанция на обекта \emph{RM} създадена с изчислените параметри. 

\begin{lstlisting}
001 > require './rm.rb'
    => true 
014 >   code = RM.detecting(5, 16)
    => RM(2, 5) 
015 > code.notation
    => [32, 16, 8] 
016 > code.corrects
    => 3 
017 > code.detects
    => 7 
018 >
\end{lstlisting}

\section{Задача 3}
За да решим задачата, създаваме клас \emph{ReedCoder} с поле \emph{code}, съдържащо кода на Рид-Малер, използван от кодера. Инстанция на този клас може да се създаде чрез извикване на метода \emph{ReedCoder\#new(code)}.

\subsection{Кодиране на информационен вектор}
Методът \emph{ReedCoder\#encode\_vector(vector)} изчислява кодовия вектор $c$ чрез умножение на информационния вектор $v$  с пораждащата матрица $G$ на използвания код на Рид-Малер $c = v \times G$.

\begin{lstlisting}
001 > require './reed_coder.rb'
       => true 
002 > require './rm.rb'
    => true 
003 > code = RM.new 1, 4
    => RM(1, 4) 
004 > coder = ReedCoder.new code
    => #<ReedCoder:0x00000002734ab8 @code=RM(1, 4)> 
005 > coder.encode_vector [1, 0, 1, 1, 0]
    => Vector[1, 1, 0, 0, 0, 0, 1, 1, 1, 1, 0, 0, 0, 0, 1, 1] 
006 > coder.decode_vector [1, 1, 0, 0, 0, 0, 1, 1, 1, 1, 0, 0, 0, 0, 1, 1]
    => Vector[1, 0, 1, 1, 0] 
007 >
\end{lstlisting}

\subsection{Декодиране на пристигнал вектор}
Методите \emph{ReedCoder\#decode\_vector(v)} и \emph{ReedCoder\#decode\_matrix(m)} декодират съответно получен вектор и матрица от получени вектори, като \emph{\#decode\_vector(v)} реализира несистематичния декодер на Рид, а \emph{\#decode\_matrix(m)} се свежда до него.
\begin{lstlisting}
001 > require './rm.rb'
    => true 
002 > require './reed_coder.rb'
    => true 
003 > v1 = [1,0,0,0,0,1,0,1,1,1,1,0,0,1,0,1]
004 > v2 = [1,0,0,1,0,1,1,0,0,1,1,0,1,0,0,1]
005 > v3 = [1,1,1,0] * 4
006 > code = RM.new 1, 4
    => RM(1, 4) 
007 > coder = ReedCoder.new code
    => #<ReedCoder:0x00000002734ab8 @code=RM(1, 4)> 
008 > coder.decode_vector v1
    => Vector[1, 0, 1, 1, 1] 
009 > coder.decode_vector v2
    => Vector[1, 1, 1, 1, 1] 
010 > coder.decode_vector v3
    => Vector[1, 0, 0, 0, 0]
011 >
\end{lstlisting}

\section*{Изходен код}

\lstinputlisting[language=Ruby]{polynom.rb}
\lstinputlisting[language=Ruby]{rm.rb}
\lstinputlisting[language=Ruby]{reed_coder.rb}

\section*{Спецификация на тестовете}
\begin{lstlisting}
Polynom
  gives you a vector of its coefficients
  only works with binary coefficients
  thorws an error when a vector of an incorrect size is given
  only accepts 2^n coefficients
  returns a correct value when given variables values to substitute with
  can tell its power
  constructs a correct argument table
  calculates a truth vector correctly
  construction
    can be done by passing a vector of coefficients
    can be done by passing a truth vector

ReedCoder
  construction
    can be done by passing an RM code
  encoding
    can be done to a vector
    can be done to a matrix
  decoding
    can be done to a vector
    can be done to a matrix
    should adjust the input vector accordingly

RM
  returns its length
  returns its dimension
  returns its minimum distance
  returns its standard code notation
  returns its information rate
  returns how many errors it can correct
  returns how many errors it can detect
  calculates the parameters of codes that can correct the passed number of errors
  calculates the perameters of codes that can detect the passed number of errors
  construction
    raises an error when a code does not exist for the given arguments
    cunstructs the generating matrix correctly

Finished in 0.0279 seconds
27 examples, 0 failures

\end{lstlisting}



\end{document}
